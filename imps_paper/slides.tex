%%% Uncomment the following for normal slide show
\documentclass[10pt,serif,professionalfont]{beamer}

%%% or uncomment this for handouts
% \documentclass[handout,ignorenonframetext]{beamer}

%%% or uncomment this for the article version
% \documentclass[11pt]{article}
% \usepackage{beamerarticle}

%% Based on a TeXnicCenter-Template by Tino Weinkauf.
%%%%%%%%%%%%%%%%%%%%%%%%%%%%%%%%%%%%%%%%%%%%%%%%%%%%%%%%%%%%%

%\usepackage[utf8]{inputenc}
\usepackage[ansinew]{inputenc}

\usepackage[english]{babel}
%\usepackage[bibnewpage]{apacite}
%\usepackage{natbib}

\usepackage{fancyhdr}
\usepackage{setspace}
\usepackage{indentfirst}
\usepackage{booktabs}

\usepackage{amsmath, amsbsy, amssymb}

\usepackage{graphics}
\usepackage{graphicx}
\usepackage{subfig}
\usepackage{caption}
\usepackage{float}
\usepackage{rotating}
\usepackage{pdflscape}
\usepackage{caption}

\usepackage{soul} %for highlighting for mark?
\usepackage{color}
%\sethlcolor{white}




%%%%%%%%%%%%%%%%%%%%%%%%%%%%
%%% Beamer Stuff
%%%%%%%%%%%%%%%%%%%%%%%%%%%%

\let\Tiny=\tiny

\mode<article>
{
  \usepackage{fullpage}
  \usepackage{pgf}
  \usepackage{hyperref}
  \setjobnamebeamerversion{example.beamer}
}

\mode<presentation>
{
  % \usetheme{Dresden}
  % \usetheme{Marburg}
  % \usetheme{Hannover}
  \usetheme{Singapore}
  % \useoutertheme{smoothbars}
  % \useoutertheme{infolines}
  \usecolortheme{seagull}
  \setbeamercovered{invisible}
  \setbeamertemplate{navigation symbols}{}%remove navigation symbols
%  \setbeamertemplate{footline}[page number]
  \setbeamertemplate{footline}[frame number]
}

\mode<handout>
{
%%% In handout mode give the individual pages a light grey background
\setbeamercolor{background canvas}{bg=black!5}
%%% Put more than one frame on each page to save paper.
\usepackage{pgfpages}
\pgfpagesuselayout{4 on 1}[letterpaper,border shrink=3mm, landscape]
% \pgfpagesuselayout{2 on 1}[letterpaper,border shrink=5mm, portrait]
% \setbeameroption{show notes}
}
% \usepackage[latin1]{inputenc}

\setbeamertemplate{itemize items}[default]
\setbeamertemplate{enumerate items}[default]
\setbeamertemplate{frametitle continuation}[from second]

\setbeamercovered{transparent}

%\usepackage{appendixnumberbeamer}

\usepackage[T1]{fontenc}
\usepackage[osf]{mathpazo}
%\linespread{1.05}         % Palatino needs more leading (space between lines)

%\usepackage{pxfonts}
%\usepackage{eulervm}


\usepackage{outlines}
\usepackage{multirow}

\renewcommand*{\thefootnote}{\fnsymbol{footnote}}

\title{Applying the Axioms of Additive Conjoint Measurement to a Hierarchy of Latent Variable Models}

\author{Ronli Diakow\inst{1}\footnote[frame]{Authors are listed in alphabetical order.} \and Benjamin Domingue\inst{2}\footnotemark[1] \and David Torres Irribarra\inst{3}\footnotemark[1]}

\subject{Tenable Assessment}

\date{International Meeting of the Psychometric Society \\ July 25, 2013}

\institute[]{
  \inst{1} New York University \and 
  \inst{2} University of Colorado at Boulder \and
  \inst{3} University of California, Berkeley}

\begin{document}

\frame{\maketitle}
%it would be great if someone could make the title page look nice

\section{Two methods}
\begin{frame}
    \frametitle{Score scales and latent structure}

    \begin{outline}
        \1 We want to appropriately characterize psychological and educational variables. 
            \2 Are differences between respondents categorical?  ordered?  distances?  
            \2 How can we determine the (``correct'') generating model?  
        
        \vspace{0.25cm}    
        
        \1 This issue has both theoretical and practical importance.  
            \2 Foundation of many debates in policy and substantive research
            \2 Any decision matters for subsequent inferences
    \end{outline}
    
\end{frame}

\begin{frame}
    \frametitle{A tale of two methods}

    \begin{outline}
        \1 Joint work based on two previous lines of research:
        \vspace{0.1cm}
            \2 Torres Irribarra and Diakow -- Can we determine the structure (qualitative, ordinal, interval) of a latent variable by model fit comparisons within a hierarchical model framework that places progressively more stringent monotonicity and scale assumptions on the latent variable?
            
            \vspace{0.25cm}
            
            \2 Domingue -- Do data possess sufficient structure, as judged by the axioms of conjoint measurement, to yield interval scales?
    \end{outline}

\end{frame}

%monotonicity and scale
\frame{

    \frametitle{Organizing models by their latent structure}

    Two kinds of restrictions that a model places on the latent variable:
    \begin{description}
        \item[\textbf{Monotonicity},] which implies that the probability of a correct response is a non-decreasing function of increasing proficiency and/or is a non-increasing function of increasing difficulty.
        \item[\textbf{Scale},] which implies that the differences between persons (and items) are quantitative (i.e. interval) in nature.  
    \end{description}
    
    %aloud: later, see how this [monotonicity] is related to the cancelation axioms

}


\frame{

    \frametitle{A hierarchy of latent variable models}

        \centering \includegraphics[width=0.75\textwidth]{./figs/Structure3.pdf} 

}


%models and equations
\frame[allowframebreaks]{

    \frametitle{Six latent variable models}

    \begin{enumerate}
    \parskip 9pt
        \item Unconstrained Latent Class Model (\textsc{un}):
            \begin{equation*} \label{eqn:lcabeta}
                \mathrm{logit} [ Pr(x_{ic} = 1 | c)] = \mathrm{logit} [\pi_{i|c}] = \beta_{ic}
            \end{equation*}

        \item Ordered Latent Class Model with Class Monotonicty (\textsc{mon}):
            \begin{eqnarray*} \label{eqn:monbeta}
               \mathrm{logit} [ Pr(x_{ic} = 1 | c)] = \mathrm{logit} [\pi_{i|c}] = \beta_{ic}, \\
               \beta_{ic} \leq \beta_{ic'} \mbox{ for all $c < c'$ and for all $i$} \nonumber
            \end{eqnarray*}

        \item Ordered Latent Class Model with Invariant Item Ordering (\textsc{iio}):
            \begin{eqnarray*} \label{eqn:iiobeta}
               \mathrm{logit} [ Pr(x_{ic} = 1 | c)] = \mathrm{logit} [\pi_{i|c}] = \beta_{ic}, \\
               \beta_{ic} \leq \beta_{i'c} \mbox{ for all $i < i'$ and for all $c$} \nonumber
            \end{eqnarray*}

        \item Ordered Latent Class Model with Double Monotonicity (\textsc{dm}):
            \begin{eqnarray*} \label{eqn:dmbeta}
               \mathrm{logit} [ Pr(x_{ic} = 1 | c)] = \mathrm{logit} [\pi_{i|c}] = \beta_{ic}, \\
               \beta_{ic} \leq \beta_{ic'} \mbox{ for all $c < c'$ and for all $i$} \nonumber, \\
               \beta_{ic} \leq \beta_{i'c} \mbox{ for all $i < i'$ and for all $c$} \nonumber
            \end{eqnarray*}

        \item Located Latent Class Model or Latent Class Rasch Model (\textsc{lcr}):
            \begin{equation*} \label{eqn:LCR}
                \mathrm{logit} [ Pr(x_{ic} = 1 | \theta_c, \delta_i)] = \theta_c - \delta_i
            \end{equation*}

        \item Rasch Model (\textsc{rm}):
            \begin{equation*}
                \label{eqn:RSH}
                \mathrm{logit} [ Pr(x_{ip} = 1 | \theta_p, \delta_i)] = \theta_p - \delta_i
            \end{equation*}
    \end{enumerate}

}


\begin{frame}
    \frametitle{Additive Conjoint Measurement (\textsc{acm})}

    \begin{outline}
        \1 Axioms of additive conjoint measurement: 
            \2 Describe conditions that need to be satisfied for an interval scale to exist for an attribute
            \2 Provide a means of testing the hypothesis that data is consistent with an interval scale  
            \2 Are difficult to verify given (likely) measurement error  
            
        \vspace{0.25cm}
        
        \1 Focus on the cancelation axioms
    \end{outline}

\end{frame}

\begin{frame} 
    \frametitle{Cancelation axioms}
    
    Consider a $3 \times 3$ probability matrix formed by the selection of 3 person abilities and 3 item difficulties:
    \[
    \begin{array}{ccc}
      x_{11} & x_{12} & x_{13}  \\
      x_{21} & x_{22} & x_{23} \\
      x_{31} & x_{32} & x_{33} \\
    \end{array}
    \]
    
    \begin{outline}
        \1 Single cancelation: Rows (people) or columns (items) can be consistently ordered.  
            \2 Implies that the major (left-leaning) diagonal is ordered.  
    
        \vspace{0.1cm}
    
        \1 Examples: 
            \2 Rows/People: If $x_{11}<x_{21} \text{, then } x_{12}<x_{22} \text{ \& } x_{13}<x_{23}$.
            \2 Columns/Items: If $x_{11}<x_{12} \text{, then } x_{21}<x_{22} \text{ \& } x_{31}<x_{32}$.
    \end{outline}

\end{frame}

\begin{frame} 
    \frametitle{Cancelation axioms}

    Consider a $3 \times 3$ probability matrix formed by the selection of 3 person abilities and 3 item difficulties:
    \[
    \begin{array}{ccc}
      \cdot&x_{12} &x_{13}  \\
      x_{21}&\cdot&x_{23} \\
      x_{31} & x_{32}&\cdot\\
    \end{array}
    \]

    \begin{outline}
        \1 Double cancelation: Additive constraints 
            \2 Imposes some order on the (very messy) minor (right-leaning) diagonal.  
        
        \vspace{0.1cm}
            
        \1 Two forms:
            \2 If $x_{21}<x_{12} \text{ \& } x_{32}<x_{23} \text{ then } x_{31}<x_{13}$.
            \2 If $x_{21}>x_{12} \text{ \& } x_{32}>x_{23} \text{ then } x_{31}>x_{13}$.

    \end{outline}

\end{frame}

\begin{frame}
    \frametitle{Applying the axioms of \textsc{acm}}

    \begin{outline}
        \1 Domingue expanded and improved a Bayesian method for checking the single and double cancelation constraints
            \2 Accounts for measurement error in the observed proportions

        \vspace{0.25cm}
        
        \1 The method implemented in the R package ConjointChecks: 
            \2 Estimate the posterior for the probability of a correct response within each cell using relevant cancelation constraints to form the jumping distribution
            \2 Check if the observed proportions correct fit within the estimated 95\% credible interval

    \end{outline}

\end{frame}

\begin{frame}

    \frametitle{The framework and the axioms}

        %temporal until I prepare the new figure
        \centering \includegraphics[width=0.75\textwidth]{./figs/Structure3.pdf} \\

\end{frame}


\section{Study design}

\begin{frame}
    \frametitle{Hypotheses}
    
    %first just show table, then add sentences
    
    %\parskip 12pt

    \begin{center}
    \begin{tabular}{lccc}
    \toprule
     \multirow{3}{*}{Model} & \multicolumn{3}{c}{Satisfies cancelation axiom} \\ \cmidrule(lr){2-4}
                & \multicolumn{2}{c}{Single} & \multirow{2}{*}{Double} \\ \cmidrule(lr){2-3}
                  & Rows       & Columns    &            \\
    \midrule
     \textsc{un}  &            &            &            \\
     \textsc{mon} & \checkmark &            &            \\
     \textsc{iio} &            & \checkmark &            \\
     \textsc{dm}  & \checkmark & \checkmark &            \\
     \textsc{lcr} & \checkmark & \checkmark & \checkmark \\
     \textsc{rm}  & \checkmark & \checkmark & \checkmark \\
    \bottomrule
    \end{tabular}
    \end{center}

    \uncover<2->{
    Expected order for number of violations of cancelation axioms: 
    \vspace{-0.2cm}
    \begin{center}
    \textsc{un} > ( \textsc{mon} $\stackrel{?}{\sim}$ \textsc{iio} ) > \textsc{dm} > ( \textsc{lcr} $\stackrel{?}{\sim}$ \textsc{rm} )
    \end{center}
    }

    \uncover<3->{
    This should lead to fairly straightforward criteria to recover the generating latent structure.  
    }
    \uncover<4>{\alert{However...}}
    
    %aloud: thought this would be straightforward, but ran into deeper issues: sensitivity to ``mesh''; number of different variables that we are trying to disentangle.  

\end{frame}

\begin{frame}
    \frametitle{Simulation design and analysis}

        \begin{outline}
        \1 Generate data under each of six models 
        \renewcommand{\outlineii}{enumerate}
            \2 Use original data from Torres Irribarra and Diakow
                \3 5000 people in 2-6 classes
                \3 10 items
                \3 30 replications per model
            \2 Simulate new data 
                \3 1000 people in 6 classes
                \3 50 items in 6 groups
                \3 50 replications per model
        
        \vspace{0.25cm}
        
        \1 Check for violations of the cancelation axioms using ConjointChecks
        \renewcommand{\outlineii}{itemize}
            \2 Double cancelation and each single cancelation separately

        \vspace{0.25cm}

        \1 Record the \% of (weighted) violations for each model
    \end{outline}

\end{frame}


\section{Results}

\begin{frame}
    \frametitle{Simulation 1: Results for single cancelation}
        \framesubtitle{Person ordering}

    \centering \includegraphics[width=0.75\textwidth, clip, trim = 0 4.5in 0 0]{./figs/boxplots_single.pdf}

\end{frame}

\begin{frame}
    \frametitle{Simulation 1: Results for single cancelation}
        \framesubtitle{Item ordering}

    \centering \includegraphics[width=0.75\textwidth, clip, trim = 0 0 0 4.5in]{./figs/boxplots_single.pdf}

\end{frame}

\begin{frame}
    \frametitle{Simulation 1: Results for double cancelation}

    \centering \includegraphics[width=0.75\textwidth, clip, trim = 0 0 0 4.5in]{./figs/boxplots.pdf}

\end{frame}

\begin{frame}
    \frametitle{Simulation 1: Results summary}
    
    \begin{center}
    \begin{tabular}{lccc}
    \toprule
     \multirow{3}{*}{Model} & \multicolumn{3}{c}{Average percentage of violations} \\ \cmidrule(lr){2-4}
                & \multicolumn{2}{c}{Single} & \multirow{2}{*}{Double} \\ \cmidrule(lr){2-3}
                  & Rows       & Columns    &            \\
    \midrule
     \textsc{un}  & 0.XX & 0.XX & 0.XX \\
     \textsc{mon} & 0.XX & 0.XX & 0.XX \\
     \textsc{iio} & 0.XX & 0.XX & 0.XX \\
     \textsc{dm}  & 0.XX & 0.XX & 0.XX \\
     \textsc{lcr} & 0.XX & 0.XX & 0.XX \\
     \textsc{rm}  & 0.XX & 0.XX & 0.XX \\
    \bottomrule
    \end{tabular}
    \end{center}
    
\end{frame}

\begin{frame}
    \frametitle{Simulation 2: Single cancelation}
        \framesubtitle{Person ordering}

    \centering \includegraphics[width=0.75\textwidth]{./figs/violations_columns_weighted.pdf}

\end{frame}

\begin{frame}
    \frametitle{Simulation 2: Single cancelation}
        \framesubtitle{Item ordering}

    \centering \includegraphics[width=0.75\textwidth]{./figs/violations_rows_weighted.pdf}

\end{frame}

\begin{frame}
    \frametitle{Simulation 2: Double cancelation}

    [include figure]

\end{frame}

\begin{frame}
    \frametitle{Simulation 2: Results summary}

    \begin{center}
    \begin{tabular}{lccc}
    \toprule
     \multirow{3}{*}{Model} & \multicolumn{3}{c}{Average percentage of violations} \\ \cmidrule(lr){2-4}
                & \multicolumn{2}{c}{Single} & \multirow{2}{*}{Double} \\ \cmidrule(lr){2-3}
                  & Rows       & Columns    &            \\
    \midrule
     \textsc{un}  & 0.XX & 0.XX & 0.XX \\
     \textsc{mon} & 0.XX & 0.XX & 0.XX \\
     \textsc{iio} & 0.XX & 0.XX & 0.XX \\
     \textsc{dm}  & 0.XX & 0.XX & 0.XX \\
     \textsc{lcr} & 0.XX & 0.XX & 0.XX \\
     \textsc{rm}  & 0.XX & 0.XX & 0.XX \\
    \bottomrule
    \end{tabular}
    \end{center}

\end{frame}

\section{Discussion}

\begin{frame}
    \frametitle{Discussion}

    \begin{outline}
        \1 Person monotonicity versus item ordering
            \2 Precision and/or aggregation 
            \2 Formally, people and items are symmetric; in the real world, we rarely treat them symmetrically.  

        \vspace{0.25cm}

        \1 Reconsidering double cancelation
            \2 Checking for double cancelation did not add stringency
            \2 Observed data is very noisy relative to differences in true probabilities
            \2 Casts doubt on the utility of this method for checking double cancelation

    \end{outline}

\end{frame}

\begin{frame}
    \frametitle{Next steps}
    
    \begin{outline}
        \1 Can we manipulate data generation such that we can accurately predict changes in the proportion of violations?
        
        \vspace{0.25cm}
        
        \1 Are the double cancelation results idiosyncratic, or do they apply more generally?  
       
    \end{outline}

\end{frame}

\frame[contactinfo]{

    \begin{center}

    \Large Applying the Axioms of Additive Conjoint Measurement to a Hierarchy of Latent Variable Models

    \vspace{10mm}

    \small

    \begin{tabular}{rl}

        ben.domingue@gmail.com & \textsc{Benjamin Domingue} \\
        dti@berkeley.edu       & \textsc{David Torres Irribarra} \\
        rdiakow@berkeley.edu   & \textsc{Ronli Diakow}

    \end{tabular}

    \end{center}

    \vspace{10mm}

%    \scriptsize
%    \hspace{5mm} Torres Irribarra, D., and Diakow, R.  (2011, July).  Model Selection for Tenable Assessment: Selecting a Latent Variable Model by Testing the Assumed Latent Structure.  Paper presented at the 76th Annual and 17th International Meeting of the Psychometric Society, Hong Kong.


}


%back-up slides
\appendix

%fix so slide numbering ignores back-up slides
\newcounter{finalpage}
\setcounter{finalpage}{\value{page}}
\newcounter{finalframe}
\setcounter{finalframe}{\value{framenumber}}
\setbeamertemplate{footline}{}

\begin{frame}

    \centering Appendix Slides

\end{frame}

%fix so slide numbering ignores back-ups
\setcounter{page}{\value{finalpage}}
\setcounter{framenumber}{\value{finalframe}}

\end{document}


\begin{frame}
    \frametitle{}

\end{frame} 